% !TeX root = ../main.tex

\chapter{Conclusion} \label{chapter:conclusion}

Bla bla...

% Sum up the project and results

\section{Disussion}

% Discuss the resutls? What is good or bad? And why?


% wahl der richtigen loss function spielt eine sehr wichtige rolle. dabei kommt es immer auf die gegebenheiten der daten an. die 3 verschiedenen datensets haben das gezeigt, da jedes set mit einer anderen loss-function am besten abgeschnitten hat.



\section{Future work}

% Network might be a good initialization for activity-recog/class:
% --> "5.3 Unsupervised Learning as an Optimization Strategy" in
%     http://www.iro.umontreal.ca/~lisa/pointeurs/TR1312.pdf
%     Also: Unsupervised Learning with LSTMs paper
%     But: hätte den Rahmen dieser Arbeit gesprungen. 

% use the encoder + learned rep ==> as pre-trained model supervised learning tasks

Bla bla...

\begin{itemize}
\item Adversarial Neural Networks
	\begin{itemize}
	\item Integrate our advanced architecture in the DCGAN framework
	\end{itemize}
\item Multi-scales approaches
\item Additional loss in feature space (not just image space)
\item Gradient/Cell-clipping?
\item ...
\end{itemize}