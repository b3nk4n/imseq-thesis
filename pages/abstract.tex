\chapter{\abstractname}

Deep networks are becoming central in several areas of computer vision and image processing task. But while there has been a lot of research regading classification of images or videos, future frame prediction is still a rarely studied approach. However, many applications could make good use of the knowledge regarding the next frame of a video in pixel-space, such as video compression or autonomous agents in robotics that have to act in natural environments. In fact learning how to forecast the future of an image sequence requires the system to understand and efficiently encode the conent and dynamics for a certein period of time. Furthermore, it is viewed as a promising avenue where even supervised tasks could benefit from, because large datasets of labeled video data are limited and very hard to obtain. Therefore, an overview of existing approaches that cover future frame prediction is given and a new network model is presented which utilzes recent advances from deep learning research. The proposed architecture is based on the recurrent decoder-endocder framework with convolutional cells which allow to preserve spatio-temporal correlations of the data. Driven by perceptual motivated objective functions and new recurrent learning strategy, it is able to outperform many existing approaches in several types of videos, even though it is trained for fewer iterations and contains fewer model parameters.