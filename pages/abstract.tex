\chapter{\abstractname}

Deep networks are becoming central in several areas of computer vision and image processing tasks. While there has been a lot of research regarding the classification of images and videos, future frame prediction is still a rarely studied approach. However, many applications could make good use of the knowledge regarding the next frame of a video in pixel-space. Examples include video compression and autonomous agents in robotics that have to act in natural environments. In fact, learning how to forecast the future of an image sequence requires the system to understand and efficiently encode the content and dynamics for a certain period of time. Furthermore, it is viewed as a promising avenue which even supervised tasks could benefit from, because large datasets of labeled video data are limited and very hard to obtain. Therefore, an overview of existing approaches that cover future frame prediction is given and a new network model is presented which utilizes recent advances from deep learning research. The proposed architecture is based on the recurrent decoder-encoder framework, with convolutional cells which allows the preservation of spatio-temporal correlations of data. Driven by perceptual motivated objective functions and a new recurrent learning strategy, it is able to outperform existing approaches in several video content types. All this can be achieved with fewer training iterations and model parameters.



