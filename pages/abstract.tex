\chapter{\abstractname}

Neural networks are becoming central in several areas of computer vision. While there has been a lot of studies regarding the classification of images and videos, future frame prediction is still a rarely investigated approach. However, many applications could make good use of the knowledge regarding the next frame of a video in pixel-space. Examples include video compression and autonomous agents in robotics that have to act in natural environments. In fact, learning how to forecast the future of an image sequence requires the system to understand and efficiently encode the content and dynamics for a certain period of time. It is viewed as a promising avenue in which even supervised tasks could benefit from, since labeled video data is limited and hard to obtain. Therefore, an overview of scientific advances covering future frame prediction is given and proposes a recurrent network model which utilizes recent techniques from deep learning research. The presented architecture is based on the recurrent decoder-encoder framework with convolutional cells, which allow the preservation of spatio-temporal data correlations. Driven by perceptual motivated objective functions and a modern recurrent learning strategy, it is able to outperform existing approaches with respect to future frame generation in several video content types. All this can be achieved with fewer training iterations and model parameters.



